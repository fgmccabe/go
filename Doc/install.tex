\chapter{Installing \go}
\label{install}

This appendix gives instructions on installing \go and \april on a Unix-based system. It is possible to install and use both on a Windows system, under Cygwin; however this is beyond the scope of this chapter to explain how.

\emph{Warning:}
\begin{quote}
The information in this appendix is subject to change and may be different in your situation.
\end{quote}

\section{Getting \go}
The installation of \go requires the installation of three packages: \q{ooio}, \april and \go proper. For a number of reasons, \go is currently only distributed as source.

The \q{ooio} library is a basic library that supports many functions of the other two packages.

The \april language system is a full programming language. It is used only to compile \go programs, however.

The \go language system is the complete engine, compiler and documentation.

A good place to get the source tarballs for these packages is
\begin{alltt}
http://homepage.mac.com/frankmccabe
\end{alltt}

Assuming that you have the files \q{ooio-\em{date}.tgz}, \q{april-\em{date}.tgz} and \q{go-\em{date}.tgz} then the process involves first of all compiling and installing \q{ooio}, then \april, and then doing a similar job for \go.

\section{Installation directory}
The systems \q{ooio}, \april and \go are designed to be installed by default in the directories \q{/opt/ooio}, \q{/opt/april} and \q{/opt/go} respectively. It is possible to set up both to install in different locations; for example in your home directory. However, we \emph{do not} recommend installing either in the `standard' installation directories \q{/usr/bin} and/or \q{/usr/local/bin}. However, it is certainly possible to install \emph{links} to the \april/\go compilers and run-time systems in those locations.

The installation directory is important as it includes a number of files that are important for the smooth running of both \april and \go.

You can make the appropriate directories in the default locations using:
\begin{alltt}
\% mkdir /opt/nar /opt/april /opt/go
\% chown <yourUserName> /opt/nar /opt/april /opt/go
\end{alltt}

\section{Building the \q{ooio} library}
Unpack the \q{ooio-\em{date}.tgz} file and enter the \q{ooio} top-level directory. The \q{ooio} library is set up to use the \q{configure} script -- which was automatically generated using a combination of \q{automake} and \q{autoconf}. 

To configure \q{ooio} run the \q{autogen.sh} script:
\begin{alltt}
ooio \% ./configure [--prefix=\emph{dir}]
\ldots
\end{alltt}
It is only necessary to supply the \q{--prefix} argument if \q{/opt/ooio} is not the preferred installation directory.

Once configured, compile the library using the \q{make} command:
\begin{alltt}
ooio \% make all
\end{alltt}

If you own the target directory, then you can simply:
\begin{alltt}
ooio \% make install
\end{alltt}
to install it; otherwise, try:
\begin{alltt}
ooio \% sudo make install
\end{alltt}
remembering, of course, that the password the system will ask for is your's, not that of root.

\section{Building \april}
The process to build and install \april is similar to that for the \q{ooio} library; with the additional wrinkle of dealing with a non-standard location for the \q{ooio} library.

Unpack the \q{april.tgz} tarball and enter the \q{april5} directory. 

To configure \april run the \q{configure} script:
\begin{alltt}
april5 \% ./configure [--prefix=\emph{dir}] [--with-ooio=\emph{ooioDir}]
\ldots
\end{alltt}
It is only necessary to supply the \q{--prefix} argument if \q{/opt/april} is not the preferred installation directory, and the \q{--with-ooio} argument is only needed if you installed \q{ooio} in a non-standard place.

Once configured, compile \april and install it using the \q{make} command:
\begin{alltt}
april5 \% make all install
\end{alltt}

\subsection{Paths}
Since \april (and \go) are designed to install in their own directories, it is useful to extend the \q{path} environment to include them. You can do this in your \q{.bash\_login} file in your home directory\note{The process for doing this if you use a different shell will be similar.}:
\begin{alltt}
export PATH=\$PATH:/opt/april/bin:/opt/go/bin
\end{alltt}

\section{Building \go}
The process for setting up \go is exactly analogous to that for \april; excepting that if you have installed \april in a non-standard directory, you need to inform \go's configuration script:
\begin{alltt}
go \% ./configure[--with-april=\emph{dir}] [--with-ooio=\emph{ooioDir}]
\end{alltt}

Making \go is also similar:
\begin{alltt}
go \% make
\end{alltt}
There is one additional step in building \go: checking that everything is running Ok. The command:
\begin{alltt}
go \% make check
\end{alltt}
will check out quite a few of the features of \go. If this test does not terminate normally then is there is likely a problem that needs to be sorted out.

Assuming that the test works out, install \go:
\begin{alltt}
go \% make install
\end{alltt}

\section{Setting up EMACS}
\label{install:emacs}
The \go (and \april) systems come with special modes for the GNU EMACS editor.\note{Editor does not do justice to EMACS, it is more like a language/editor/operating system; if a little old-fashioned.}

To access the EMACS modes for \go and \april, you will need to modify your \q{.emacs} file. Just edit that file (in EMACS of course), and make sure that the following lines:
\begin{alltt}
(add-to-list 'load-path "/opt/go/share/emacs/site-lisp")
(add-to-list 'load-path "/opt/april/share/emacs/site-lisp")

;;; April mode
(autoload 'april-mode "april")
(setq auto-mode-alist 
  (cons '("\bsl\bsl.ap\bsl\bsl{}|.ah\$" . april-mode) auto-mode-alist))
(add-hook 'april-mode-hook 'turn-on-font-lock)

;;; Go! mode
(autoload 'go-mode "go")
(setq auto-mode-alist 
  (cons '("\bsl\bsl.\bsl\bsl(go\bsl\bsl|gof\bsl\bsl|gh\bsl\bsl)\$" . go-mode) 
    auto-mode-alist))
(add-hook 'go-mode-hook 'turn-on-font-lock)
\end{alltt}
alltt in there somewhere.

With \go mode turned on, EMACS knows how to indent \go programs according to the informal style formatting rules (very similar to those used in this book).

\section{\go Reference}
The \go reference manual will be located in the file:
\begin{alltt}
/opt/go/Doc/go-ref.pdf
\end{alltt}


