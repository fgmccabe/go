\chapter{Grammar rules}
\label{grammars}

\index{logic grammars}
\index{grammar rule}
A \go grammar consists of grammar rules of the form:
\begin{alltt}
N,T\sub1,\ldots,T\subn --> R\sub1,\ldots,R\sub{k}
\end{alltt}
where \qe{N} is a non-terminal, \qe{T\subi} are all terminals and \qe{R\sub{j}} are either terminals or non-terminals.

The stream of data that is processed by a grammar rule is typically a \q{string}. However, in general, the stream may be represented by any kind of list.

\section{Grammar conditions}
\index{logic grammars!basic conditions}

\subsection{Terminal grammar condition}
\label{grammar:terminal}

\index{logic grammars!terminal}
An expression enclosed in a list represents a \emph{terminal grammar condition}:
\begin{alltt}
[\emph{Exp}]
\end{alltt}
For grammar rules over strings, a \q{string} literal may act as a terminal grammar condition. The special case of the empty list, or empty string, is often used to denote an empty grammar condition.

\subsection{Non-terminal grammar call}
\label{grammar:nonterminal}

\index{logic grammars!non terminal}
A \emph{non-terminal grammar call} is of the form:
\begin{alltt}
\emph{nt(Args)}
\end{alltt}
this denotes a `call' to another grammar program.

The default mode for passing arguments to a grammar non-terminal is \emph{bidirectional} -- arguments are unified rather than matched.

\subsection{Class relative grammar call}
\label{grammar:dot}
\index{logic grammars!class relative non terminal}
The class relative variant of the non-terminal invokes a grammar defined within a class:
\begin{alltt}
\emph{O}.Nt(A\sub1,\ldots,A\subn)
\end{alltt}

\subsection{Equality condition}
\label{grammar:equality}

\index{grammar!equality condition}
\index{type inference!equality definition}
An equality definition grammar condition:
\begin{alltt}
\emph{Ex\sub1} = \emph{Ex\sub2}
\end{alltt}
has no effect other than to ensure that two terms are equal.

\subsection{Inequality condition}
\label{grammar:notequality}

\index{inequality}
\index{grammar!inequality}
The \q{!=} grammar condition is written:
\begin{alltt}
\emph{T\sub1} != \emph{T\sub2}
\end{alltt}
this is satisfied if the two expressions are \emph{not} unifiable.

\subsection{Query condition}
\label{grammar:goal}

\index{logic grammars!goal condition}
A \emph{query condition} is written:
\begin{alltt}
\{ \emph{Goal} \}
\end{alltt}
a query condition represents a predicate to be verified as part of the parsing process. Grammar query conditions do not `consume' any of the input.

\subsection{Disjunction grammar condition}
\label{grammar:disjunction}

A grammar \emph{disjunction} is written:
\begin{alltt}
( \emph{G\sub1} | \emph{G\sub2} )
\end{alltt}
A grammar disjunction succeeds if either arm of the disjunction is able to parse the input stream.

\subsection{Conditional grammar condition}
\label{grammar:conditional}

\index{logic grammars!conditional}
A \emph{conditional} grammar condition is written:
\begin{alltt}
(\emph{T}?\emph{G\sub1}|\emph{G\sub2})
\end{alltt}
If the grammar condition qe{T} succeeds, then the grammar \emph{G\sub1} is used to parse the input, otherwise \emph{G\sub2} is used. Note that \qe{G\sub1} sees the input after \qe{T}; however, the \qe{G\sub2} branch must parse the entire input.

\subsection{Negated grammar condition}
\label{grammar:negation}

\index{logic grammars!negated condition}
A \emph{negated} grammar condition is written:
\begin{alltt}
\nasf{} \emph{G}
\end{alltt}
where \qe{G} is a grammar condition. A negated grammar condition succeeds if \qe{G} is not able to parse the input.

\subsection{Iterated grammar}
\label{grammar:iterator}
\index{Iterated grammar condition}
\index{logic grammar!iteration}

The iterator grammar condition is written:
\begin{alltt}
\emph{G} * \emph{E} \uphat \emph{L}
\end{alltt}
This grammar succeeds if the grammar \emph{G} successfully parses any number of times. The result is returned in \emph{L} -- which consists of a list constructed from \emph{E}.

\subsection{Error handler}
\label{grammar:errorhandler}

\index{error handling!in logic grammars}
\index{logic grammars!error handling}
An \q{onerror} grammar condition takes the form:
\begin{alltt}
\emph{G} onerror (\emph{P\sub1} --> \emph{G\sub1} | \ldots{} | \emph{P\subn} --> \emph{G\subn})
\end{alltt}
An \q{onerror} grammar condition has the same meaning as the condition \emph{G}; unless a run-time problem arises. In this case, the first clause in the handler that matches with the raised error is the one that is entered. 

\subsection{Raise exception}
\label{grammar:raise}

\index{logic grammars!raise exception}
\index{error handling!exceptions in logic grammars}
The \q{raise} exception grammar condition is written:
\begin{alltt}
raise \emph{Ex}
\end{alltt}
This does not parse any input; it terminates processing of the input and raises the exception \qe{Ex}. 
\begin{aside}
By explicitly raising and handling exceptions in grammar rules we can add error handling and recovery to a grammar.
\end{aside}

\subsection{End of file}
\label{grammar:eof}

\index{logic grammars!end of file condition}
The \q{eof} grammar condition:
\begin{alltt}
eof
\end{alltt}
is satisfied only at the end of the input.