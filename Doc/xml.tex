\chapter{XML processing}
\label{xml}

\go provides a simple library for processing XML documents. This library provides programs for `grabbing' an XML document, parsing it into a DOM-like structure and displaying it. Processing XML documents is the foundation for many other forms of data processing.\footnote{This version of the XML parser is not a complete XML 1.0 compliant parser: it does not understand external DTDs, nor does it understand all of XML 1.0's features. This may be extended in a future release.}

\go's XML processing is based on string processing. I.e., a typical mode of operation is to first of all `grab' an XML document in a string -- \q{list[char]} -- and then parse it using \q{xmlParse}. Conversely, to display an XML document, we first of all convert it to a string, and then write it out on the appropriate channel.

To use \go's XML processing facilities, it is necessary to access the \q{go.xml} package:
\libsynopsis{go.xml}

\section{\go's XML document type}
\index{XML document type structure}

The \q{xmlDOM} type provides the foundation for \go's XML processing. It is defined as:
\begin{alltt}
xmlDOM \impl showable.
xmlDOM \impl \{ 
  pickElement:[symbol]\funarrow{}xmlDOM. 
  elementPresent:[symbol]{}.
  hasAtt:[symbol,string]\{\}.
  pickAtt:[symbol]=>string.
  pickText:[symbol,string]\funarrow{}string.
  xml:[]\funarrow{}string.
\}.
\end{alltt}
The \q{xmlAttr} type encapsulates attributes associated with the \q{xmlDOM} entity (especially the \q{xmlElement} constructor):
\begin{alltt}
  xmlAttr ::= xmlAtt(symbol,string).
\end{alltt}

An \q{xmlDOM} term corresponds to the XML infoset view of an XML document. A node has one of three forms: an \q{xmlElement}, an \q{xmlText} element or an \q{xmlPI} processing instruction. Of these, the most complex is the \q{xmlElement} structure.

\begin{aside}
Note that \q{xmlDOM} is a sub-type of the \q{showable} type, which itself is in the \q{go.showable} package. However, it is not required for any package that is simply \emph{using} the \q{xmlDOM} type to also import the \q{go.showable} package.
\end{aside}

\begin{description}
\item[pickElement]
The \q{pickElement} function returns the first child element whose name is that given.

\q{pickElement} raises a \q{'eNOTFND'} exception if the element is not present.

\item[elementPresent]
The \q{elementPresent} predicate is satisfied if the named element is present as a child.

\item[hasAtt]
The \q{hasAtt} predicate is satisfied if the named attribute is present.

\item[pickAtt]
The \q{pickAtt} function returns the value of the named attribute if present. Note that the XML specification permits multiple values for an attribute of an element; however, the \q{pickAtt} function returns an arbitrary member if there is more than one.

If the attribute is not present, then an \q{'eNOTFND'} exception will be raised.

\item[xml]
The \q{xml} function returns an XML string representation of the \q{xmlDOM}. 

Note that while most of the methods are not meaningful for all \q{xmlDOM} constructores, the \q{xml} function is.
\end{description}


\subsection{xmlText}
\label{xml:xmltext}
\index{xmlText@\q{xmlTEXT} xmlDOM constructor}
\classsynopsis{xmlText}{[string]}{xmlDOM}
Text appearing in an XML document is collected in an \q{xmlText} term. Note that the parser removes empty text elements during the parse, and strips off leading and trailing white space from other text elements.

The parser supports standard entities (see Section~\vref{xml:entities}); which are substituted for in the body of the \q{xmlText} term.

Note that \q{xmlText} does \emph{not} support many of the methods in the \q{xmlDOM} type -- for example, invoking the \q{pickElement} function will result in an \q{'eINVAL'} exception being raised. The \q{xmlText} constructor is a state-free constructor, and the expectation is that normally it will be matched against rather than used.

\subsection{xmlPI}
\label{xml:xmlPI}
\index{xmlPI@\q{xmlPI} xmlDOM constructor}
\classsynopsis{xmlPI}{[string]}{xmlDOM}
Processing instructions appearing in the XML document are returned in an \q{xmlPI} term.

\subsection{xmlElement}
\label{xml:xmlElement}
\index{xmlElement@\q{xmlElement} xmlDOM constructor}
\classsynopsis{xmlElement}{[symbol,list[xmlAttr],list[xmlDOM]]}{xmlDOM}
A tagged element is represented as an \q{xmlElement} term. This has three arguments, corresponding to the tag, the list of attributes and the list of child elements.

The parser automatically converts local tag names into their fully expanded form when there is a name-space declaration; including the default namespace declaration.

For example, the XML fragment:
\begin{alltt}
<foo xmlns="http:myNameSpace#" id="bar">
  <subfoo/>
</foo>
\end{alltt}
is represented, using \q{xmlDOM} terms, as:
\begin{alltt}
xmlElement('http:myNameSpace#foo',
  [xmlAtt('http:myNameSpace#id',"bar")],
  [xmlElement('http:myNameSpace#subfoo',[],[])])
\end{alltt}

\subsection{xmlAtt -- attributes}
\label{xml:xmlAtt}
\index{xmlAtt@\q{xmlAtt} xmlDOM attribute constructor}
\classsynopsis{xmlAtt}{[symbol,string]}{xmlAttr}
Atributes in an \q{xmlElement} are represented using the \q{xmlAtt} term. Each attribute \q{xmlAtt} term has two arguments: the attribute name and the value of the attribute -- as a string.

The \go xml parser expands standard entities in attribute values.

\subsection{Standard entities}
\label{xml:entities}
The \go parse (and display) functions are aware of a restricted set of standard entities. It is not possible to declare new entity definitions. The standard entities that the parser is aware of are shown in Table~\ref{xml:standardentities}:
\begin{table}[h]
\begin{center}
\begin{tabular}{|l|c|}
\hline
Entity&Definition\\
\hline
\tt \&amp;&\tt\&\\
\tt \&lt;&\tt<\\
\tt \&gt;&\tt>\\
\tt \&apos;&\tt'\\
\tt \&excl;&\tt!\\
\tt \&quot;&\tt"\\
\hline
\end{tabular}
\end{center}
\caption{\go standard entities}\label{xml:standardentities}
\end{table}

\section{Parsing XML documents}
\label{xml:parse}

There are two programs in the standard xml library for parsing documents: \q{grabURL} which is a convenience function for accessing the contents of a URI and \q{xmlParse} which is a grammar program for parsing strings.

\subsection{grabXML}
\synopsis{grabXML}{[string,string]\funarrow{}(string,xmlDOM)}

The \q{grabXML} function takes two \q{string} arguments: a `base' url (\param{B}), a `request' url (\param{U}) and returns a pair -- the fully resolved url and the contents of the document found at that url, parsed as an \q{xmlDOM} document. The request url may be relative, in which case it is interpreted relative to the base url \param{B}.

\paragraph{Error exceptions}\footnote{See Chapter~\vref{errorcodes} for the definition of the standard error codes}
\begin{description}
\item[\constant{'eNOTFND'}]
It was not possible to access the document at the resolved location.
\end{description}

\subsection{xmlParse}

\synopsis{xmlParse}{[xmlDOM-]\grarrow{}string}

The \q{xmlParse} grammar parses a string which contains an XML document and returns the \q{xmlDOM} structure corresponding to the parse.

The \q{xmlParse} parser does not validate XML documents, nor does it process DTDs. However, it is `namespace' aware: documents can be processed with namespace declarations and recognized properly.

The parser is not currently a full XML parser; Table~\vref{xml:xml} is an enumeration of the features of XML and the extent to which they are supported.
\begin{table}[h]
\begin{center}
\begin{tabular}{|l|c|}
\hline
XML 1.0 Feature&Supported\\
\hline
Attributes&yes\\
Character references&yes\\
Comments&yes\\
Conditional sections&no\\
DTD&no\\
Document declaration&no\\
Empty Tags&yes\\
Entities&predefined only\\
Namespaces&yes\\
PCDATA&yes\\
Processing Instructions&yes\\
Tags&yes\\
Validation&no\\
Well-formedness&partial\\
\hline
\end{tabular}
\end{center}
\caption{XML 1.0 features}\label{xml:xml}
\end{table}
Although the parser records processing instructions, there are no PIs that it recognizes directly.

\section{Displaying XML documents}

\subsection{xmlDisplay - show XML structure}
\label{xml:display}
\synopsis{xmlDisplay}{[xmlDOM]\funarrow{}string}

The \q{xmlDisplay} function returns a string XML representation of an \q{xmlDOM} term. It uses the local tag and attribute names in constructing the displayed string.

\q{xmlDisplay} is `entity' aware: characters in the values of attributes and in \q{xmlText} elements that require representing as elements in order to conform to XML are handled correctly.

\q{xmlDisplay} is an extremely simple generator, optimized for performance rather than elegance of output. In particular, it does not `pretty print' the XML structure into an easily readable form.

\section{Miscelleneous functions}

\subsection{\q{hasNameSpace} -- Look for a name space}
\synopsis{hasNameSpace}{[xmlDOM,string]\{\}}

The \q{hasNameSpace} predicate is true of an \q{xmlElement} if has an \q{xmlns} attribute -- i.e., if its default namespace is known. The second argument is the name space.

This predicate is equivalent to:
\begin{alltt}
hasNameSpace(X,N) :- X.hasAtt('xmlns',N).
\end{alltt}

