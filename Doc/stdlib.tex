\chapter{Standard Library}
\label{stdlib}
The standard libraries contain suites of programs that are useful in many different situations. 

For many of the functions in this library, it is not necessary to explicitly \q{import} the corresponding packages. In fact, these programs are defined in the standard library \q{go.stdlib} -- which is always automatically loaded for all programs.

However, some functions are in separate packages and will need to be explicitly \q{import}ed. All the packages in the standard library are part of the \q{go} package domain, and thus require an \q{import} statement of the form:
\begin{alltt}
\emph{packageName}\{
  import go.\emph{library}.
  \ldots
\}
\end{alltt}
Where a function requires a specific package to be \q{import}ed, the description of the function explains which package to load.

\section{List manipulation}
\go has a simple set of functions that can be used to manipulate lists. These are essentially culled from experience of what list functions are most used; of these list append (\q{<>}) is probably the most heavily used list function.

\subsection{\function{<>} -- List append}
\label{stdlib:append}
\synopsis{<>}{[list[t],list[t]]\funarrow{}list[t]}
\index{<>@\q{<>} function}
\index{Appending lists!with \q{<>} function}

The \q{<>} polymorphic function appends its two arguments together: both arguments being lists of the same type. The definition of list append is quite short:
\begin{alltt}
[] <> X => X.
[E,..X] <> Y => [E,..X<>Y].
\end{alltt}
As with many of the programs in the standard library, the \emph{name} of the function is also an operator; allowing the use of infix notation in list append expressions.

\subsection{\q{append} -- List append predicate}
\label{stdlib:app}
\synopsis{append}{[list[t],list[t],list[t]]\{\}}
\index{append@\q{append} predicate}
\index{Appending lists!with \q{append} predicate}

The \q{<>} list append function is convenient to use for the large majority of cases where it is used to concatenate two lists together. However, the traditional \prolog \q{append} predicate can be used for many additional things -- such as splitting lists, searching lists and so on. It is defined as:
\begin{alltt}
append([],X,X).
append([E,..X],Y,[E,..Z]):-append(X,Y,Z).
\end{alltt}

\subsection{\function{listlen} -- Length of a list}
\label{stdlib:listlen}
\index{length of a list}
\index{{\tt listlen} function}
\synopsis{listlen}{[list[t]]\funarrow{}integer)}

The \q{listlen} function counts the length of the list \param{L}. \q{listlen} is defined as though by:
\begin{alltt}
listlen(L) => len(L,0).

private len:[list[t],integer]=>integer.
len([],C) => C.
len([\_,..L],C) => len(L,C+1).
\end{alltt}

\subsection{\function{in} -- List membership}
\label{stdlib:in}
\synopsis{in}{[T,list[T]]\{\}}
\index{in@\q{in} predicate}
\index{Test for list membership}

The \q{in} standard predicate is true if the left hand argument term is on the list in the right hand argument.

The definition of \q{in} is:
\begin{alltt}
X in [X,..\_].
X in [\_,..Y] :- X in Y.
\end{alltt}
Note that unification is used to determine element equality, and that an element may be on a list more than once: the \q{in} predicate will find subsequent entries on backtracking.
\begin{aside}
If you look for the definition of \q{in} in the \q{go.stdlib} package source you will not find it. This is because \q{in} has such a deep relationship to many of \go's features that the compiler generates specific versions of \q{in} whenever it is used.
\end{aside}

\subsection{\function{reverse} -- List reverse}
\label{stdlib:rev}
\synopsis{reverse}{[list[t]]\funarrow{}list[t]}
\index{reverse@\q{reverse} a list function}
\index{Function to reverse lists}

The \q{reverse} polymorphic function reverses its argument. This is \emph{not} `naive reverse'; \q{rev} is a linear complexity function. The definition of \q{rev} uses an auxilliary function to accomplish list reversal:
\begin{alltt}
reverse(L) => rv(L,[]).

private rv:[list[t],list[t]]=>list[t].
rv([],R) => R.
rv([E,..M],R) => rv(M,[E,..R]).
\end{alltt}

\subsection{\function{nth} -- Nth element of a list}
\label{misc:nth}
\index{Nth element of a list}
\index{nth@\q{nth} function}

\synopsis{nth}{[list[T],integer]\funarrow{}t}

The \q{nth} function returns the n$^{th}$ element of a list. The definition of \q{nth} is:
\begin{alltt}
nth(L,\_)::var(L)=>exception error("nth",'eINVAL').
nth([E,..\_],1) => E.
nth([\_,..L],N) => nth(L,N-1).
nth([],\_) => raise error("nth",'eNOTFND').
\end{alltt}

\subsection{\q{front} -- the front portion of a list}
\label{stdlib:front}
\index{front@\q{front} of list function}
\index{Get front portion of list}
\synopsis{front}{[list[t],integer]\funarrow{}list[t]}

The \q{front} function returns the front portion of a list. It is defined as though by:
\begin{alltt}
front([],\_)=>[].
front([E,..L],C)::C>0 => [E,..front(L,C-1)].
front(\_,\_) => [].
\end{alltt}

\subsection{\q{tail} -- the tail portion of a list}
\label{stdlib:tail}
\index{tail@\q{tail} of list function}
\index{Get back portion of list}
\synopsis{tail}{[list[t],integer]\funarrow{}list[t]}

The \q{tail} function returns the tail portion of a list, i.e., the last N elements of the list. It is defined as though by:
\begin{alltt}
tail(L,C)::append(_,Tl,L), listlen(Tl,C) => Tl.
\end{alltt}
Note that its actual definition is significantly more efficient than this specification!

\subsection{\q{drop} -- drop n elements from a list}
\label{stdlib:drop}
\index{drop@\q{drop} from list function}
\synopsis{drop}{[list[t],integer]\funarrow{}list[t]}

The \q{drop} function drops the first N elements from a list. It is defined as:
\begin{alltt}
drop:[list[t],integer]=>list[t].

drop([],_)=>[].
drop(L,0) => L.
drop([_,..L],N) => drop(L,N-1).
\end{alltt}


\subsection{\function{iota} -- List construction}
\label{stdlib:iota}
\index{{\tt iota} function}
\index{construct a list of integers}

\synopsis{iota}{[integer,integer]\funarrow{}list[integer]}

The \q{iota} function constructs a list of integers from \param{X} to \param{Y} inclusive. The definition of \q{iota} is:
\begin{alltt}
iota(N,M)::N>M=>[].
iota(N,M)=>[N,..iota(N+1,M)].
\end{alltt}

\section{Set manipulation}
These programs support sets, represented as lists. Sets are not a fully first-class structure in \go: there is no built in set data type. However, it is perhaps more useful to provide a suite of functions that implement set-like semantics over lists. In addition to these functions, \go also has two specific operators for constructing set-like lists: the bag-of expression (see section~\vref{expression:bagof}) and the bounded set expression (see section~\vref{expression:bounded}).

These set functions are accessed via the \q{setlib} library:

\libsynopsis{go.setlib}

\subsection{\texorpdfstring{\function{\union} -- }{}Set union}
\label{stdlib:union}
\synopsis{\union}{[list[t],list[t]]\funarrow{}list[t]}

The \q{\union} (pronounced `union') function unions two sets -- represented as lists -- together. Note that the \union function assumes that both arguments are already sets -- i.e., do not contain duplicate elements.

The definition of \q{\union} is:
\begin{alltt}
[]\union{}X=>X.
[E,..X]\union{}Y :: E in Y => X\union{}Y.
[E,..X]\union{}Y => [E,..X\union{}Y].
\end{alltt}

\subsection{\texorpdfstring{\function{\intersection} -- }{}Set intersection}
\label{stdlib:intersection}
\synopsis{\intersection}{[list[t],list[t]]\funarrow{}list[t]}

The \q{\intersection} (pronounced `intersection') function intersects two sets -- represented as lists -- together. Note that the \intersection function assumes that both arguments are already sets -- i.e., do not contain duplicate elements.

The definition of \q{\intersection} is:
\begin{alltt}
[]\intersection{}\_=>[].
[E,..X]\intersection{}Y :: E in Y=> [E,..X\intersection{}Y].
[E,..X]\intersection{}Y => X\intersection{}Y.
\end{alltt}

\subsection{\texorpdfstring{\function{\difference} -- }{}Set difference}
\label{stdlib:difference}
\synopsis{\difference}{[list[t],list[t]]\funarrow{}list[t]}

The \q{\difference} (pronounced `difference') function forms the set difference of two sets -- represented as lists -- together. Note that the \difference function assumes that both arguments are already sets -- i.e., do not contain duplicate elements.

The definition of \q{\difference} is:
\begin{alltt}
[]\difference{}\_=>[].
[E,..X]\difference{}Y :: \nasf{}E in Y => [E,..X\difference{}Y].
[E,..X]\difference{}Y => X\difference{}Y.
\end{alltt}

\subsection{\texorpdfstring{\function{subset} -- }{}Subset predicate}
\label{setlib:subset}
\synopsis{subset}{[list[t],list[t]]\{\}}

The \q{subset} predicate is true if every element of the first list is also an element of the second.

For example:
\begin{alltt}
subset([1,34],[1,2,34,10,34,21,-3])
\end{alltt}
is satisfied, whereas
\begin{alltt}
subset([1,3,35],[1,2,34,10,34,21,-3])
\end{alltt}
is not, as \q{35} is not an element of the second argument.

The definition of \q{subset} is:
\begin{alltt}
subset(X,Y) :-
  E in X *> E in Y
\end{alltt}

